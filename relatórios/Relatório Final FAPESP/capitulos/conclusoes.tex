\chapter{Considerações Finais}
\label{cp:conclusoes}

Experimentos controlados envolvem o controle de certos parâmetros para medir a influência desses em variáveis dependentes, que variam de acordo com o objeto de estudo. 

Embora sugerido na literatura, o protocolo de experimentação não é explicitado em Pacotes de Laboratório e, consequentemente, não é visível ao experimentador nem ao replicador. Desse modo, pesquisadores encontram dificuldades para estabelecer o plano de execução de um experimento, assim como, compreender o conteúdo de pacotes de laboratório para a replicação de experimentos~\cite{Wohlin2012}.

Neste trabalho foi proposta a construção de uma interface que viabilize a concepção e modelagem do protocolo do experimento, assim como o armazenamento deste conteúdo em um Pacote de Laboratório.

\section{Cronograma Inicial}
O cronograma de atividades foi organizado de acordo com a Tabela ~\ref{tab_cronograma}. As atividades foram divididas mensalmente, com início em dezembro de 2016. Nessa tabela as letras 'x' correspondem às atividades previstas, 'X' às atividades já realizadas.

\begin{itemize}
\item \textit{Atividade 1}: Revisão Bibliográfica (e da tecnologia a ser utilizada);
\item \textit{Atividade 2}: Definição dos requisitos da camada de interface;
\item \textit{Atividade 3}: Desenvolvimento da camada de interface;
\item \textit{Atividade 4}: Avaliação da interface desenvolvida;
\item \textit{Atividade 5}: Relatório Parcial;
\item \textit{Atividade 6}: Relatório Final.

\end{itemize}


\begin{table}[htbp]
\centering 
\caption{Cronograma de Atividades}
\label{tab_cronograma}
\small
\begin{tabular}
{|c|c|c|c|c|c|c|c|c|c|c|c|c|} \cline{2-13}
\multicolumn{1}{c|}{}&\multicolumn{12}{c|}{\textbf{Período}}
 \\
\cline{2-13}
\multicolumn{1}{c|}{}&\multicolumn{1}{c|}{\textbf{2016}} &\multicolumn{11}{c|}{\textbf{2017}} \\
\hline Atividades & Dez & Jan & Fev & Mar & Abr & Mai & Jun & Jul & Ago & Set & Out & Nov \\
\hline         1  & X   & X   &  X  &  X  &     &     &     &     &     &     &     &     \\
\hline         2  &     &     &  X  &  X  &     &     &     &     &     &     &     &     \\
\hline         3  &     &     &     &  X  &  X  &  X  &  X  &  X  &  X  &     &     &     \\
\hline         4  &     &     &     &     &     &     &     &     &  X  &  X  &  X  &  X  \\
\hline         5  &     &     &     &     &     &  X  &     &     &     &     &     &     \\
\hline         6  &     &     &     &     &     &     &     &     &     &     &  X  &  X  \\
\hline
\end{tabular}
\normalsize
\end{table}

\section{Atividades Realizadas}

Conforme estabelecido pelo cronograma inicial, as atividades previstas para a primeira parte do projeto, anteriores a entrega do relatório científico parcial, tinham como propósito o embasamento teórico dentro do contexto da Engenharia de Software Experimental sobre experimentos controlados, o uso de pacotes de laboratório para transferência de dados entre grupos de pesquisa. Em seguida, iniciou-se o estudo sobre modelos de processo de negócio e suas notações para representação de modelos gráficos, em específico a BPMN.

Em seguida, a partir dos requisitos da camada de interface para modelagem, foi iniciada a implementação desta ferramenta de modo a viabilizar a construção dos modelos de processo de negócio com todos os elementos previstos pela documentação da notação em sua versão atual.

Após a elaboração desta interface de modelagem, foi iniciada a construção da camada de persistência dos modelos de processo, assim como a recuperação e armazenamento destes em documentos de forma permanente. De modo análogo também foi construída uma camada de persistência e uma interface de dados para obtenção dos conteúdos relativos aos experimentos executados e armazenados segundo o modelo proposto pela ontologia \textit{ExperOntology} a partir da ferramenta \textit{OntoExpTool}. Desse modo, a partir da leitura de um pacote de laboratório é possível incorporar o modelo processo de negócio respectivo ao experimento, e assim, incorporando o protocolo de experimentação.

Em seguida, foram estabelecidas avaliações para a detecção de inconsistências tanto na ferramenta quanto na representação da notação, e desta forma, suas devidas correções. Para esta tarefa foram obtidos modelos de processos presentes na documentação oficial da notação BPMN, de forma a detectar se os modelos criados a partir ferramenta foram construídos de forma valida quanto ao conjunto de itens e seus relacionados entre elementos.

Também foram construídos alguns modelos de diagramação de forma a representar os protocolos de estudos experimentais já previamente executados para avaliar se a notação apresentada supre todas as necessidades encontradas. Em relação a instanciação de novos pacotes de laboratório, foi executado o processo de engenharia reversa a partir de um pacote de pré-existente foi elaborado o respectivo protocolo de experimental (plano de execução) e por fim, a adição do modelo de processo a nova instância do pacote de laboratório. 

Por fim, foi iniciada a construção do presente relatório científico final.

\section{Contribuições}

A principal contribuição do presente trabalho está em fornecer uma ferramenta que permita a construção e a visualização do plano de execução (protocolo) de estudos experimentais, assim como a integração desses dados a um pacote de laboratório e a vinculação do conjunto de artefatos presentes  em experimentos às suas respectivas atividades representadas por um modelo de processo de negócio.

Apesar de ter como foco a concepção do plano de execução de experimentos controlados, não foi detectada nenhuma limitação na construção de planos de execução também aos modelos de estudos experimentais (pesquisa de opinião e estudos de caso) apresentados na Seção~\ref{sect:processoexperimentacao}. Um restrição consiste na vinculação de informações do experimento, por adotar o modelo proposto na ontologia \textit{ExperOntology}. Porém, essa restrição não está relacionado a utilização de modelos de processo de negócio, mas sim pelo modelos de dados adotado.

Adicionalmente, a ferramenta também permite a manipulação de pacotes de laboratório e protocolos de experimentação sob o modelo não-relacional de dados orientado a documentos, utilizando a linguagem XML. Além da legibilidade deste formato, fornece maior portabilidade entre sistemas de software facilitando a integração de novos componentes por se tratar de um formato padrão na troca de dados, em detrimento a diversos sistemas de gerenciamento de banco de dados proprietários e com diversas restrições presentes no modelo relacional.


\section{Trabalhos Futuros}

Do ponto de vista da ferramenta desenvolvida, os seguintes aprimoramentos são desejáveis:

\begin{itemize}

\item Aprimoramento de alguns recursos gráficos da ferramenta, para uma melhor usabilidade na modelagem de processos de negócio, com a adição de recursos com redimensionamento para alguns elementos do tipo \textit{container}, assim como indicativos de seleção de itens na criação de transições.

\item Integração da camada de interface de modelagem de processo de negócio com a ferramenta de instanciação de pacotes de laboratório, viabilizando a adição de elementos a descrição do experimento por meio dos modelos gráficos \textit{OntoExpTool}.

\end{itemize}

Na perspectiva da Engenharia de Software Experimental, o próximo passo a ser feito no contexto do Grupo de Pesquisa, é apresentado a seguir como uma continuação para este Projeto de Iniciação Científica.

%%%%%%%%%%%%%%%%%%%%%%%%%%%%%%%%%%%%%%
%%%%%%%%%%%%%%%%%%%%%%%%%%%%%%%%%%%%%%
%%%%%%%%%%%%%%%%%%%%%%%%%%%%%%%%%%%%%%

\section{Proposta de Renovação}

\subsection{Contexto, Motivação e Justificativa}

Dentro do grupo de pesquisa existem experimentos conduzidos sem a definição de seus protocolos (planos de execução) utilizando a versão atual -- modelos BPMN. 
A versão atual da ferramenta, com a interface implementada neste projeto de Iniciação Científica, permite a modelagem de protocolos de experimentos usando BPMN. 

A elaboração de modelos de processo de negócio para a representação de protocolos de experimentação visa a suprir a ausência de uma visão completa sobre o plano de execução do experimento. Porém não há evidências de que o uso da notação gráfica do BPMN auxilia a construção e o entendimento, tanto quantitativa quanto qualitativa, na compreensão de experimentos e da evolução do conhecimento armazenado em pacotes de laboratório.

É preciso ter evidências dos benefícios trazidos pelo uso de BPMN no entendimento de um Pacote de Laboratório para poder indicar seu uso como uma ferramenta para a transferência de pacotes de laboratório, intra e inter-grupos. E isso justifica a proposta de continuação deste projeto de Iniciação Científica. 


%%%%=====================================================================================================
\subsection{Formulação do Problema e Objetivo do Projeto}\label{cap:objetivos}

Como exposto, o principal problema consiste na falta de padronização na organização de pacotes de laboratório, de modo a auxiliar seu entendimento. A \textit{ExperOntology} foi proposta para organizar os dados, e agora a BPMN foi implementada como uma ferramenta (camada de interface) para modelar o protocolo do experimento. Assim, o problema é avaliar (vantagens e desvantagens, se houver) do uso de BPMN como ferramenta para expor o protocolo experimental, visando a facilitar seu entendimento. 

Assim, o objetivo principal consistem em avaliar o uso da BPMN na organização e no entendimento de protocolos experimentais. Como questões de pesquisa em Iniciação Científica, tem-se: a utilização da notação auxilia a representação do protocolo do experimento? a notação facilita a identificação e a compreensão de suas atividades e artefatos utilizados?

\subsection{Metodologia e Plano de Trabalho}

Inicialmente, serão instanciados pacotes de laboratório referentes a experimentos controlados executados no âmbito do grupo de pesquisa -- registrados com a versão textual da ferramenta \textit{OntoExpTool} (sem BPMN). Esses novos pacotes de laboratório serão construídos a partir da interface de modelagem de protocolos de experimentação já implementados. Com o término desta atividade, os experimentos já executados terão seus protocolos modelados em BPMN.  

Após a criação dos protocolos em BPMN, será feita uma revisão deles pelos autores dos experimentos originais para detectar possíveis inconsistências e incoerências. Na impossibilidade de um autor participar, o orientador revisará os protocolos. 

Uma vez aprovados pelos autores originais, os protocolos expressos em BPMN serão avaliados por terceiros. Será feito um experimento, no qual um protocolo (modelo BPMN de um experimento) será fornecido para um avaliador, que deverá descrever o experimento para verificar se, de fato, entendeu seu conteúdo. Autores de um experimentos atuarão como avaliadores de outro experimento, além de alunos do Programa de Mestrado em Ciência da Computação da UNESP (os alunos de mestrado farão tal avaliação do protocolo como parte de uma disciplina).

A última atividade consiste em analisar os resultados obtidos para que seja escrito um artigo científico.

Paralelamente às tarefas citadas anteriormente, serão redigidos o relatório parcial e final, conforme exigido. 
\begin{table}[htbp]
\centering \caption{Cronograma de Atividades}\label{tab_cronograma2}
%\small
\begin{tabular}
{|c|c|c|c|c|c|c|c|c|c|c|c|c|} \cline{2-10}
\multicolumn{1}{c|}{}&\multicolumn{9}{c|}{\textbf{Período em meses}}
 \\
\cline{2-10}
\hline Atividades & 1   & 2   & 3   & 4   & 5   & 6   & 7   & 8   & 9   \\
\hline         1  & X   & X   &     &     &     &     &     &     &     \\
\hline         2  &     &     &  X  &     &     &     &     &     &     \\
\hline         3  &     &     &     &     &  X  &  X  &     &     &     \\
\hline         4  &     &     &     &     &     &     &  X  &  X  &     \\
\hline         5  &     &     &     &  X  &     &     &     &     &     \\
\hline         6  &     &     &     &     &     &     &     &  X  & X   \\
\hline
\end{tabular}
\normalsize
\end{table}
%%%%%%%%%%%%%%
Na Tabela~\ref{tab_cronograma2} é apresentado o cronograma proposto com a duração expressa em meses, sendo que as atividades descritas foram agrupadas como segue: 

\begin{enumerate}
\item Empacotamento dos experimentos;
\item Revisão dos pacotes de laboratório;
\item Avaliação dos pacotes de laboratório;
\item Análise de resultados e elaboração do artigo científico;
\item Relatório Parcial;
\item Relatório Final.
\end{enumerate}




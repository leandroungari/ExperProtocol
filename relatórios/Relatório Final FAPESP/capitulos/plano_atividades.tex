\chapter{Plano inicial e atividades realizadas}
\label{ch:plano_atividades}

\section{Cronograma Inicial}
O cronograma de atividades foi organizado de acordo com a Tabela ~\ref{tab_cronograma}. As atividades foram divididas mensalmente, com início em dezembro de 2016. Nessa tabela as letras 'x' correspondem às atividades previstas, 'X' às atividades já realizadas e os símbolos '-' às atividades em execução.

\begin{itemize}
\item \textit{Atividade 1}: Revisão Bibliográfica (e da tecnologia a ser utilizada);
\item \textit{Atividade 2}: Definição dos requisitos da camada de interface;
\item \textit{Atividade 3}: Desenvolvimento da camada de interface;
\item \textit{Atividade 4}: Avaliação da interface desenvolvida;
\item \textit{Atividade 5}: Relatório Parcial;
\item \textit{Atividade 6}: Relatório Final.

\end{itemize}


\begin{table}[htbp]
\centering 
\caption{Cronograma de Atividades}
\label{tab_cronograma}
\small
\begin{tabular}
{|c|c|c|c|c|c|c|c|c|c|c|c|c|} \cline{2-13}
\multicolumn{1}{c|}{}&\multicolumn{12}{c|}{\textbf{Período}}
 \\
\cline{2-13}
\multicolumn{1}{c|}{}&\multicolumn{1}{c|}{\textbf{2016}} &\multicolumn{11}{c|}{\textbf{2017}} \\
\hline Atividades & Dez & Jan & Fev & Mar & Abr & Mai & Jun & Jul & Ago & Set & Out & Nov \\
\hline         1  & X   & X   &  X  &  X  &     &     &     &     &     &     &     &     \\
\hline         2  &     &     &  X  &  X  &     &     &     &     &     &     &     &     \\
\hline         3  &     &     &     &  X  &  X  &  X  &  X  &  X  &  X  &     &     &     \\
\hline         4  &     &     &     &     &     &     &     &     &  -  &  -  &  -  &  -  \\
\hline         5  &     &     &     &     &     &  X  &     &     &     &     &     &     \\
\hline         6  &     &     &     &     &     &     &     &     &     &     &  X  &  X  \\
\hline
\end{tabular}
\normalsize
\end{table}

\section{Atividades Realizadas}

Conforme estabelecido pelo cronograma inicial, as atividades previstas para a primeira parte do projeto, anteriores a entrega do relatório científico parcial, tinham como propósito o embasamento teórico dentro do contexto da Engenharia de Software Experimental sobre experimentos controlados, o uso de pacotes de laboratório para transferência de dados entre grupos de pesquisa. Em seguida, iniciou-se o estudo sobre modelos de processo de negócio e suas notações para representação de modelos gráficos, em específico a BPMN.

Em seguida, a partir dos requisitos da camada de interface para modelagem, foi iniciada a implementação desta ferramenta de modo a viabilizar a construção dos modelos de processo de negócio com todos os elementos previstos pela documentação da notação em sua versão atual.

Após a elaboração desta interface de modelagem, foi iniciada a construção da camada de persistência dos modelos de processo, assim como a recuperação e armazenamento destes em documentos de forma permanente. De modo análogo também foi construída uma camada de persistência e uma interface de dados para obtenção dos conteúdos relativos aos experimentos executados e armazenados segundo o modelo proposto pela ontologia \textit{ExperOntology} a partir da ferramenta \textit{OntoExpTool}. Desse modo, a partir da leitura de um pacote de laboratório é possível incorporar o modelo processo de negócio respectivo ao experimento, e assim, incorporando o protocolo de experimentação.

Em seguida, foram estabelecidas avaliações para a detecção de inconsistências tanto na ferramenta quanto na representação da notação, e desta forma, suas devidas correções. Também foram construídos alguns modelos de diagramação de forma a representar os protocolos de estudos experimentais já previamente executados para avaliar se a notação apresentada supre todas as necessidades encontradas.

Por fim, foi iniciada a construção do presente relatório científico final.


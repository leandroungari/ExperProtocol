\chapter*{Resumo}

\addcontentsline{toc}{section}{Resumo}

O conjunto de dados  relativos aos procedimentos, artefatos, resultados e conclusões de um experimento controlado deve ser mantido em Pacotes de Laboratório. Contudo, há relatos na literatura de dificuldades de compreensão em relação ao plano de execução do experimento, devido à ausência de informações explícitas de sua estrutura, mesmo quando baseada na ontologia \textit{ExperOntology}. Tal fato impacta negativamente a replicação ou mesmo a condução de um experimento, não permitindo uma visão geral sobre o estudo. Nesse contexto, o presente estudo propõe a utilização de modelos de processo de negócio para a modelagem de protocolos de experimentação, utilizando a notação gráfica BPMN, e sua incorporação ao Pacote de Laboratório. Assim, neste relatório são apresentadas as atividades para adequar  uma ferramenta de modo a permitir a construção de modelos de processo de negócio para a representação do protocolo do experimento em um Pacote de Laboratório. Adicionalmente, a ferramenta contribui para a portabilidade e transferência de instâncias de Pacotes de Laboratório, facilitando a legibilidade e integração por  utilizar de um modelo não-relacional orientado a documentos (XML). Neste texto também é apresentada a proposta para renovação da bolsa de iniciação científica.
\chapter{Considerações Finais}
\label{cp:conclusoes}

Nessa primeira parte do projeto, além dos estudos teóricos, foi implementada a camada de persistência referente à primeira versão da respectiva notação, assim como classes e métodos auxiliares armazenar no formato \textit{XML}, conforme a estruturação de um pacote de laboratório. Paralelamente, tem sido desenvolvida a camada de interface para a construção dos diagramas de processo de negócio.

Para uma avaliação mais precisa do uso da notação BPM na elaboração de processo de elaboração de protocolos de experimentos, é necessária a conclusão das duas atividades citadas acima, assim como a transferência dos dados de um pacote de laboratório para a representação gráfica.

A continuidade das atividades corresponde ao encerramento da implementação da camada de persistência assim como da interface de modelagem. Em seguida, será estabelecida a integração da interface de modelagem BPM com a ferramenta \textit{OntoExpTool}, de modo a visualizar graficamente o protocolo do experimento armazenado no pacote de laboratório.

A execução das atividades do projeto está dentro do esperado, desse modo, a conclusão das atividades deve ocorrer dentro dos prazos estabelecidos, assim como a elaboração do relatório científico final deste projeto de iniciação científica.
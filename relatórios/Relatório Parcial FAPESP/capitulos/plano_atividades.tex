\chapter{Plano inicial e atividades realizadas}
\label{ch:plano_atividades}

\section{Cronograma Inicial}
O cronograma de atividades foi organizado de acordo com a Tabela ~\ref{tab_cronograma}. As atividades foram divididas mensalmente, com início em dezembro de 2016. Nessa tabela as letras 'x' correspondem às atividades previstas e as letras 'X' às atividades já realizadas.

\begin{itemize}
\item \textit{Atividade 1}: Revisão Bibliográfica (e da tecnologia a ser utilizada);
\item \textit{Atividade 2}: Definição dos requisitos da camada de interface;
\item \textit{Atividade 3}: Desenvolvimento da camada de interface;
\item \textit{Atividade 4}: Avaliação da interface desenvolvida;
\item \textit{Atividade 5}: Relatório Parcial;
\item \textit{Atividade 6}: Relatório Final.

\end{itemize}


\begin{table}[htbp]
\centering 
\caption{Cronograma de Atividades}
\label{tab_cronograma}
\small
\begin{tabular}
{|c|c|c|c|c|c|c|c|c|c|c|c|c|} \cline{2-13}
\multicolumn{1}{c|}{}&\multicolumn{12}{c|}{\textbf{Período}}
 \\
\cline{2-13}
\multicolumn{1}{c|}{}&\multicolumn{1}{c|}{\textbf{2016}} &\multicolumn{11}{c|}{\textbf{2017}} \\
\hline Atividades & Dez & Jan & Fev & Mar & Abr & Mai & Jun & Jul & Ago & Set & Out & Nov \\
\hline         1  & X   & X   &  X  &  X  &     &     &     &     &     &     &     &     \\
\hline         2  &     &     &  X  &  X  &     &     &     &     &     &     &     &     \\
\hline         3  &     &     &     &  X  & X   &  x  &  x  & x   & x   &     &     &     \\
\hline         4  &     &     &     &     &     &     &     &     & x   & x   &  x  &  x  \\
\hline         5  &     &     &     &     &     &  X  &     &     &     &     &     &     \\
\hline         6  &     &     &     &     &     &     &     &     &     &     &   x &  x  \\
\hline
\end{tabular}
\normalsize
\end{table}

\section{Atividades Realizadas}

As atividades desenvolvidas até o presente momento foram as seguintes: foi realizada uma breve revisão bibliográfica sobre Engenharia de Software Experimental, com foco em experimentos controlados e o uso de Pacotes de Laboratório para a transferência de dados entre grupos de pesquisa, considerando o conhecimento adquirido em iniciações científicas anteriores. Em seguida, foi aprofundado o estudo em relação ao uso da notação BPM e como esta poderia ser aplicada em contexto de estudos experimentais.

Atualmente, a partir dos requisitos da camada de interface para modelagem, foi iniciada a implementação desta assim como da camada de persistência.

Em relação às próximas atividades, o próximo passo consiste em finalizar a construção da camada de persistência e da interface gráfica de modelagem. Em seguida, reaizar a integração com a ferramenta \textit{OntoExpTool} e, desta forma, avaliar o uso da notação para a construção de protocolo de experimentação. Por fim, será realizada a escrita do relatório científico final.

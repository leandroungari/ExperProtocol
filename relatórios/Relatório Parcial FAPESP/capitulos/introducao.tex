\chapter{Introdução}

Este relatório tem como objetivo apresentar as atividades desenvolvidas pelo bolsista Leandro Ungari Cayres referente ao projeto  registrado junto à FAPESP com número 2016/17477-2, sob orientação do Prof. Dr. Rogério E. Garcia, período compreendido entre dezembro de 2016 a maio de 2017.

O objetivo geral deste projeto consiste em prover uma interface capaz de apresentar visualmente o protocolo de um experimento, utilizando a notação BPM (\textit{Business Process Modeling}). Ou seja, deve-se: (1) prover ao experimentador a possibilidade de planejar seu experimento utilizando a notação BPM e; (2) prover ao replicador a possibilidade de visualizar o protocolo contido no PL, também utilizando a notação BPM.

Como objetivo específico, considera-se a modificação da camada de apresentação (interface) da ferramenta \textit{OntoExpTool}, incorporando o modelo gráfico BPM à ferramenta, assim como as adequações necessárias na camada de controle. É importante ressaltar que objetiva-se constituir um sistema de software que permita a concepção e a troca de pacotes de laboratórios, para apoiar a condução de experimentos controlados.

Como questões de investigação (em nível de Iniciação Científica), tem-se: o uso de BPM contribui para a definição do protocolo de experimentos? Quais recursos devem ser utilizados para o uso da tecnologia? Como suplantar limitações para a tecnologia a ser utilizada?

Para apresentar as atividades desenvolvidas até o momento, este relatório encontra-se dividido em $5$ capítulos, além desta introdução:

\begin{itemize}

\item No capítulo \ref{cp:engenharia} é apresentada uma breve revisão sobre Engenharia de Software Experimental, com foco em experimentos controlados, para contextualização do projeto de pesquisa.

\item No capítulo \ref{cp:bpmn} é descrito um breve histórico da notação BPM, assim como, apresentando os elementos que compõem este notação.

\item No capítulo \ref{cp:prototipo} é apresentado o andamento das atividades de implementação da interface de modelagem.

\item No capítulo \ref{ch:plano_atividades} é apresentado o plano de atividades inicial para o projeto. É, também, apresentado o que fora cumprido até o presente momento, bem como as atividades que serão efetuadas a partir desta etapa.

\item Por fim, no Capítulo \ref{cp:conclusoes} são apresentadas as considerações finais do presente relatório.
\end{itemize}
     

